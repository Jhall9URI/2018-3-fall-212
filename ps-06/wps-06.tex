\documentclass[11pt]{article}

    \usepackage{listings}
    \usepackage{fancyhdr}
    \usepackage[margin=.8in]{geometry}
    \usepackage{amsmath}
    \usepackage{enumitem}
    \usepackage{hyperref}
    
    \linespread{1.2}
    \setlength{\parindent}{0pt}
    \setlength{\tabcolsep}{15pt}
    
    % ===========================================================================
    % Header / Footer
    % ===========================================================================
    
    \pagestyle{fancy}
    \lhead{\scriptsize  CSC 212: Data Structures and Abstractions - Fall 2018}\chead{}\rhead{\scriptsize Weekly Problem Set Solutions \#6}
    \lfoot{}\cfoot{\scriptsize \thepage~of~\pageref{r:lastpage}}\rfoot{}
    \renewcommand{\headrulewidth}{0.3pt}
    \renewcommand{\footrulewidth}{0.3pt}
    
    % ===========================================================================
    % ===========================================================================
    \begin{document}
    \thispagestyle{empty}
    
    % ===========================================================================
    \begin{center}
        {\Large\bf CSC 212: Data Structures and Abstractions}\\
        \medskip
        {\Large\bf Fall 2018}\\
        \medskip
        {\Large\bf University of Rhode Island}\\
        \bigskip
        {\Large\bf Weekly Problem Set \#6}
    \end{center}
    
    Due Thursday 11/1 at the beginning of class. Please turn in neat, and organized, answers hand-written on standard-sized paper \textbf{without any fringe}. At the top of each sheet you hand in, please write your name, and ID.
    
    \begin{enumerate}
        \item Implement merge sort. Your function should take an array of integers and the indices of the first and last elements in the list to sort.
        \item Implement quicksort. Your function should take an array of integers and the indices of the first and last elements in the list to sort.
        \item Define a recurrence relation for the best case for quicksort. Assume that a partition for an array of size n takes n comparisons.
        \item Solve your recurrence relation for question 3.
        \item Define a recurrence relation for the worst case for quicksort. Assume that a partition for an array of size n takes n comparisons.
        \item Solve your recurrence relation for question 5.
    \end{enumerate}
    \label{r:lastpage}
    \end{document}
        
    