\documentclass[11pt]{article}

    \usepackage{listings}
    \usepackage{fancyhdr}
    \usepackage[margin=.8in]{geometry}
    \usepackage{amsmath}
    \usepackage{enumitem}
    \usepackage{hyperref}
    \usepackage{tikz}
    \usetikzlibrary{arrows}
    \tikzset{
        treenode/.style = {align=center, inner sep=0pt, text centered,
            font=\sffamily},
        arn_n/.style = {treenode, circle, black, font=\sffamily\bfseries, draw=black,
            fill=white, text width=1.5em},
        arn_r/.style = {treenode, circle, red, draw=red,
            text width=1.5em, very thick},
        arn_x/.style = {treenode, rectangle, draw=black,
            minimum width=0.5em, minimum height=0.5em}
    }

    \linespread{1}
    \setlength{\parindent}{0pt}
    \setlength{\tabcolsep}{20pt}

    % ===========================================================================
    % Header / Footer
    % ===========================================================================

    \pagestyle{fancy}
    \lhead{\scriptsize  CSC 212: Data Structures and Abstractions - Fall 2018}\chead{}\rhead{\scriptsize Weekly Problem Set \#08}
    \lfoot{}\cfoot{\scriptsize \thepage~of~\pageref{r:lastpage}}\rfoot{}
    \renewcommand{\headrulewidth}{0.25pt}
    \renewcommand{\footrulewidth}{0.25pt}

    % ===========================================================================
    % ===========================================================================
    \begin{document}
    \thispagestyle{empty}

    % ===========================================================================
    \begin{center}
        {\Large\bf CSC 212: Data Structures and Abstractions}\\
        \medskip
        {\Large\bf Fall 2018}\\
        \medskip
        {\Large\bf University of Rhode Island}\\
        \bigskip
        {\Large\bf Weekly Problem Set \#08}
    \end{center}

    Due Thursday 11/15 before class. Please turn in neat, and organized, answers hand-written on standard-sized paper \textbf{without any fringe}. At the top of each sheet you hand in, please write your name, and ID.

    \section{K-ary Trees}
    \begin{enumerate}
        \item Draw a k-ary tree, where \verb|k=4|, after the insertion of the following elements in order: \emph{Assuming insertions are performed left to right, level by level}

        \verb|[5, 4, 6, 8, 2, 9, 10, 1]|

        \item Looking at the tree you have drawn, how many leaves and nodes are present?

        \item Examine your tree and find both the root and the node with the value 4. For both nodes, list the following attributes: depth, height of subtrees, number of siblings, number of children.

        \item Insert 6 more random elements into your tree and relist any of the above attributes that have changed.

        \item Would the structure (shape of the tree and not values of the nodes) of the k-ary tree you've drawn change at all if the elements were inserted in sorted order? Explain why or why not.
    \end{enumerate}

    \section{Doubly Linked Lists}

    \begin{enumerate}

        \item Describe an algorithm to count the number of sets of three adjacent nodes whose sum is equal to 0.

        \item For a doubly linked list with n elements, how much additional memory is going to be used compared to a singly linked list?

    \end{enumerate}

    \section{Stacks and Queues}
    \begin{enumerate}

        \item Is a linked list the best underlying structure to implement a queue with? Justify your answer.

        \item Is a linked list the best underlying structure to implement a stack with? Justify your answer.

        \item Would a stack or queue be more efficient for the following:
        \begin{enumerate}
            \item An undo button in a text editor
            \item A web server
            \item A breadth-first search
            \item A depth-first search
        \end{enumerate}
    \end{enumerate}

    \label{r:lastpage}

    \end{document}
