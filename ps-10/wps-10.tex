\documentclass[11pt]{article}

    \usepackage{listings}
    \usepackage{fancyhdr}
    \usepackage[margin=.8in]{geometry}
    \usepackage{amsmath}
    \usepackage{enumitem}
    \usepackage{hyperref}
    \usepackage{tikz}
    \usetikzlibrary{arrows}
    \tikzset{
        treenode/.style = {align=center, inner sep=0pt, text centered,
            font=\sffamily},
        arn_n/.style = {treenode, circle, black, font=\sffamily\bfseries, draw=black,
            fill=white, text width=1.5em},
        arn_r/.style = {treenode, circle, red, draw=red, 
            text width=1.5em, very thick},
        arn_x/.style = {treenode, rectangle, draw=black,
            minimum width=0.5em, minimum height=0.5em},
        node_num/.style = {treenode, circle, black, font=\sffamily\bfseries, draw=black, fill=white, text width=1.5em},
        node_highlight/.style = {treenode, circle, red, draw=red, text width=1.5em, very thick},
        node_null/.style = {treenode, rectangle, draw=white, minimum width=0.5em, minimum height=0.5em}
    }
    
    \linespread{1}
    \setlength{\parindent}{0pt}
    \setlength{\tabcolsep}{20pt}
    
    % ===========================================================================
    % Header / Footer
    % ===========================================================================
    
    \pagestyle{fancy}
    \lhead{\scriptsize  CSC 212: Data Structures and Abstractions - Spring 2018}\chead{}\rhead{\scriptsize Weekly Problem Set \#10}
    \lfoot{}\cfoot{\scriptsize \thepage~of~\pageref{r:lastpage}}\rfoot{}
    \renewcommand{\headrulewidth}{0.25pt}
    \renewcommand{\footrulewidth}{0.25pt}
    
    % ===========================================================================
    % ===========================================================================
    \begin{document}
    \thispagestyle{empty}
    
    % ===========================================================================
    \begin{center}
        {\Large\bf CSC 212: Data Structures and Abstractions}\\
        \medskip
        {\Large\bf Fall 2018}\\
        \medskip
        {\Large\bf University of Rhode Island}\\
        \bigskip
        {\Large\bf Weekly Problem Set \#10}
    \end{center}
    
    Due Thursday 12/7 before class. Please turn in neat, and organized, answers hand-written on standard-sized paper \textbf{without any fringe}. At the top of each sheet you hand in, please write your name, and ID.
    \section{2-3 Trees}
\begin{enumerate}
    \item Draw a 2-3 tree after inserting the following elements: [6, 2, 8, 5, 10, 3, 1, 7, 9, 4]

    \item What steps does a 2-3 tree search algorithm take when searching the drawn tree for 5? Assume that the lesser element of a 3-node is checked first.
\end{enumerate}

\section{Left Leaning Red-Black Trees}
\begin{enumerate}
    \item Draw a Red-Black tree after inserting the following elements: [1, 2, 3, 4, 5]
    
    \item What happens in general when inserting elements in ascending order?
    
    \item Can Red-Black Trees be represented as 2-3 trees? If so, how?

    \item Why is it advantageous to have balanced trees?

    \item True or False: as you insert nodes into a Red-Black Tree, the height is non-decreasing?
    
\end{enumerate}
    
    \label{r:lastpage}
    
    \end{document}
        
    
