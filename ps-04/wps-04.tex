\documentclass[11pt]{article}

    \usepackage{listings}
    \usepackage{fancyhdr}
    \usepackage[margin=.8in]{geometry}
    \usepackage{amsmath}
    \usepackage{enumitem}
    
    \linespread{1.3}
    \setlength{\parindent}{0pt}
    \setlength{\tabcolsep}{15pt}
    
    % ===========================================================================
    % Header / Footer
    % ===========================================================================
    \pagestyle{fancy}
    \lhead{\scriptsize  CSC 212: Data Structures and Abstractions - Spring 2018}\chead{}\rhead{\scriptsize Weekly Problem Set \#4}
    \lfoot{}\cfoot{\scriptsize \thepage~of~\pageref{r:lastpage}}\rfoot{}
    \renewcommand{\headrulewidth}{0.3pt}
    \renewcommand{\footrulewidth}{0.3pt}
    
    % ===========================================================================
    % ===========================================================================
    \begin{document}
    \thispagestyle{empty}
    
    % ===========================================================================
    \begin{center}
        {\Large\bf CSC 212: Data Structures and Abstractions}\\
        \medskip
        {\Large\bf Fall 2018}\\
        \medskip
        {\Large\bf University of Rhode Island}\\
        \bigskip
        {\Large\bf Weekly Problem Set \#4}
    \end{center}
    
    Due Thursday 10/11 at the beginning of class. Please turn in neat, and organized, answers hand-written on standard-sized paper \textbf{without any fringe}. At the top of each sheet you hand in, please write your name, and ID.
    
    \begin{enumerate}
        \item Write a recursive function that sums all of the elements of a given $n$ length array, matching this signature: \verb|int sum(int* arr, int n);|
        \item Rewrite the recursive sum function to only sum odd numbers within the array.
        \item Write a recursive function that can find the minimum of a given array, matching this signature:
        \verb|int min_array(int* arr, int n);|
        \item Reverse the elements of an array in place. Matching the following function signature:\\
        \verb|void reverse_array(int* arr, int n);|
    \item Write a function to print triangles to \verb|std::cout| that takes three positive integers: $a$, $b$, $c$ as input. The function should print the \verb|+| character $a$ times, then $a+c$ times, then $a+c+c$ times, and so on. This pattern should repeat until the line is $b$ characters long. At that point, the pattern is repeated backwards. For example calling \verb|draw_triangle(4, 7, 1)| will output: (where the dollar symbol is the bash command prompt)
    \begin{verbatim}
        ++++
        +++++
        ++++++
        +++++++
        +++++++
        ++++++
        +++++
        ++++
    \end{verbatim} 

    \item For both insertion and selection sort, describe if the algorithm is stable and if not give an example array that shows the unstable behavior.
    \end{enumerate}
    
    \label{r:lastpage}
    
    \end{document}
